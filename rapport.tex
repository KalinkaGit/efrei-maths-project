\documentclass[a4paper,12pt]{report}

% Encodage de caractères
\usepackage[utf8]{inputenc}

% Utilisation des accents
\usepackage[T1]{fontenc}

% Gestion des marges
\usepackage{geometry}
\geometry{top=2.5cm, bottom=2.5cm, left=3cm, right=3cm}

% Pour les graphiques
\usepackage{graphicx}   

% Pour les liens cliquables dans le PDF
\usepackage{hyperref}

% Pour les tableaux
\usepackage{array}

% Pour personnaliser l'en-tête et le pied de page
\usepackage{fancyhdr}

% Titre
\title{Tp : Métro - Boulot - Dodo}
\author{Hugo LEROUX, Rémi GRIMAULT, Axel JACQUET}
\date{Année : 2024 - 2025}

\begin{document}

\maketitle

\tableofcontents % Génère la table des matières

% Personnalisation de l'en-tête et du pied de page
\pagestyle{fancy}
\fancyhf{} % Efface l'en-tête et le pied de page par défaut
\fancyfoot[R]{\thepage} % Numérotation des pages à droite dans le pied de page
\renewcommand{\headrulewidth}{0pt} % Supprime la ligne dans l'en-tête
\renewcommand{\footrulewidth}{0.4pt} % Ajoute un trait dans le pied de page

% Ajoute la numérotation des pages à partir de la page 2
\setcounter{page}{2}

\chapter{Introduction}
Dans ce chapitre, nous introduirons les concepts fondamentaux nécessaires pour comprendre ce rapport. Ce document présente un exemple de page de rapport en \LaTeX{}, incluant des sections, sous-sections, tableaux et autres éléments communs.

\section{Contexte}
Le contexte de ce rapport se base sur l'utilisation de \LaTeX{} pour la création de documents structurés. Il est couramment utilisé pour la rédaction de rapports scientifiques, techniques ou académiques.

\section{Objectifs}
L'objectif de ce rapport est de démontrer les fonctionnalités de base de \LaTeX{} pour la création d'un document académique.

\chapter{Méthodologie}
Dans cette section, nous décrivons la méthodologie utilisée pour réaliser l'exemple de rapport.

\section{Collecte des données}
Les données ont été collectées via une méthode d'observation directe.

\subsection{Méthode qualitative}
Les observations ont été réalisées sur un échantillon de 50 participants. Un questionnaire a été utilisé pour recueillir les réponses.

\section{Analyse des données}
Les données ont été analysées à l'aide d'outils statistiques comme les tests de moyenne et l'analyse de variance.

\chapter{Résultats}
Les résultats montrent que l'utilisation de \LaTeX{} est bénéfique pour la rédaction de documents académiques.

\section{Tableau des résultats}
Voici un exemple de tableau pour présenter les résultats :

\begin{table}[h!]
\centering
\begin{tabular}{|c|c|c|}
\hline
\textbf{Critère} & \textbf{Résultat} & \textbf{Interprétation} \\ \hline
PER & 18.5 & Valeur acceptable \\ \hline
ROCE & 16\% & Bon rendement \\ \hline
\end{tabular}
\caption{Tableau des résultats financiers}
\end{table}

\chapter{Conclusion}
En conclusion, ce rapport a montré comment utiliser \LaTeX{} pour créer un document structuré, avec des sections, des tableaux, et des références bibliographiques.

\chapter{Références}
\begin{thebibliography}{9}
\bibitem{latex} 
Lamport, L. \textit{\LaTeX: A Document Preparation System}. Addison-Wesley, 1994.

\bibitem{latex2} 
Knuth, D. \textit{The Art of Computer Programming}. Addison-Wesley, 1968.
\end{thebibliography}

\end{document}
